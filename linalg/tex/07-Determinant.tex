\section*{7 The Determinant}

\paragraph{D 7.2.1} Given a \key{permutation}
$\sigma:\{1,\dots,n\}\to\{1,\dots,n\}$ of $n$ elements, its \key{sign}
$\sgn(\sigma)$ can be $1$ or $-1$. Then sign counts the parity of the number of
pairs of elements that are out of order (sometimes called inversions) after
applying the permutation. In other words, for all pairs
$(i,j)\in\{1,\dots,n\}\times\{1,\dots,n\}$:

\begin{equation*}
    \sgn(\sigma) =
    \begin{cases}
        1 &\text{if } \lvert \{(i,j) \, | \, i<j, \sigma(i)>\sigma(j)\} \rvert \text{ even} \\
        -1 &\text{if } \lvert \{(i,j)  \, | \, i<j, \sigma(i)>\sigma(j)\} \rvert \text{ odd}
    \end{cases}
\end{equation*}

\paragraph{D 7.2.3} Let $A\in\mathbb{R}^{\nxn}$ be a square
matrix and let $\Pi_n$ be the set of all permutations of $n$ elements. Then the
\key{determinant} $\det(A)$ is defined as:

\begin{equation*}
    \det(A)=\sum_{\sigma\in\Pi_n}\sgn(\sigma)\prod_{i=1}^{n}A_{i,\sigma(i)}
\end{equation*}

\paragraph{P 7.2.4}
\begin{enumerate}
    \item Given a permutation matrix
$P\in\mathbb{R}^{\nxn}$ corresponding to a permutation $\sigma$, then
$\det(P)=\sgn(\sigma)$. We sometimes also write $\sgn(P)$.

    \item Given a triangular (either upper or lower) matrix
$T\in\mathbb{R}^{\nxn}$, we have $\det(T)=\prod^n_{k=1} T_{kk}$.

    \item If $Q\in\mathbb{R}^{\nxn}$ is an orthogonal
matrix, then $\det(Q)=1$ or $\det(Q)=-1$.
\end{enumerate}

\paragraph{T 7.2.5} Given a matrix $A\in\mathbb{R}^{\nxn}$, we have
$\det(A^\top)=\det(A)$.

\paragraph{T 7.2.6}
\begin{enumerate}
    \item $A\in\mathbb{R}^{\nxn}$ is invertible iff $\det(A)\not=0$.

    \item Given $A,B\in\mathbb{R}^{\nxn}$, we have $\det(AB)=\det(A)\det(B)$.

    \item Given $A\in\mathbb{R}^{\nxn}$ such that $\det(A)\not=0$, then $A$ is invertible, and $\det(A^{-1})=\frac{1}{\det(A)}$.
\end{enumerate}

\paragraph{D 7.3.1} Given $A\in\mathbb{R}^{\nxn}$, for each
$1\leq i,j\leq n$ let $\mathscr{A}_{ij}$ denote the $(n-1)\times (n-1)$ matrix
obtained by removing row $i$ and column $j$ from $A$. Then we define the
\key{co-factors of $A$} as $C_{ij}=(-1)^{i+j}\det(\mathscr{A}_{ij})$.

\paragraph{P 7.3.2} Let $A\in\mathbb{R}^{\nxn}$, then for any
$1\leq i\leq n$: $\det(A)=\sum_{j=1}^{n}A_{ij}C_{ij}$.

\paragraph{P 7.3.3} Let $A\in\mathbb{R}^{\nxn}$ with
$\det(A)\not=0$ and let $C$ be the $\nxn$ matrix with the co-factors of
$A$ as entries. Then $A^{-1} = \frac{1}{\det(A)}C^\top$.

\paragraph{P 7.3.5} (\key{Cramer's rule}) Let
$A\in\mathbb{R}^{\nxn}$ such that $\det(A)\not=0$ and
$b\in\mathbb{R}^n$. Moreover, let $\mathscr{B}_j$ be the matrix obtained
from $A$ by replacing the $j$-th column of $A$ with the vector $b$. Then
the solution $x\in\mathbb{R}^n$ of $Ax=b$ is given by $x_j=\frac{\det(\mathscr{B}_j)}{\det(A)}$.
