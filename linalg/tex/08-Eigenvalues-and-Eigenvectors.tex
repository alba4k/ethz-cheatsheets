\section*{8 Eigenvalues and Eigenvectors}

\subsection{Complex Numbers}

The \key{complex numbers} are of the form $z=a+ib$ for $a\in\mathbb{R}$
and $b\in\mathbb{R}$: $\mathbb{C}=\{a+ib\mid a,b\in\mathbb{R}\}$. Hereby $i$ is
defined as $i^2=-1$.

Given $z\in\mathbb{C}$ with $z=a+ib$, we have the following notation:

\begin{enumerate}
    \item $\mathfrak{R}(a+ib) := a$
        \hfill (\key{real part} of $z$)

    \item $\mathfrak{I}(a+ib) := b$
        \hfill (\key{imaginary part} of $z$)

    \item $\lvert z\rvert := \sqrt{a^2+b^2}$
        \hfill (\key{modulus} of $z$)

    \item $\overline{a+ib} := a-ib$
        \hfill (\key{complex conjugate} of $z$)

    \item $A^H = A^* := \overline{A}^\top$
        \hfill (\key{conjugate/hermitian transpose} of $A$)

    \item $||v||^2 = v*v = \overline{v}^\top v$
        \hfill (\key{modulus} of $vn$)

    \item $\langle v, w \rangle = w^* v$
        \hfill (\key{inner/dot product} in $\mathbb{C}^n$)
\end{enumerate}

We also have $|z|^2 = z\overline{z}$, $z_1z_2=z_2z_1$, $\overline{z_1+z_2} = \overline{z_1}+\overline{z_2}$, $\frac{1}{\overline{z}} = \frac{\overline{z}}{|z|^2}$.

The natural operation of "transposing" for complex vectors and matrices is that of the "\key{conjugate transpose}" or "hermitian transpose", denoted by $A^*$, or sometimes $A^H$: $v^*=\overline{v}^\top$ and $A^*=\overline{A}^\top$.

The \key{inner product} (or dot-product) in $\mathbb{C}^n$ is given by $\langle
v,w\rangle=w^*v$.

\paragraph{T 8.1.2} (\key{Fundamental T of Algebra}) Any degree $n$
non-constant ($n\geq 1$) polynomial $P(z)$ has a zero (root)
$\lambda\in\mathbb{C}$, such that $P(\lambda)=0$.

\paragraph{C 8.1.3} Any degree $n$ non-constant ($n\geq 1$) polynomial $P(z)=\alpha_nz^n+\cdots+\alpha_1z+\alpha_0$ has $n$ zeros (roots) $\lambda_1,\dots,\lambda_n\in\mathbb{C}$, perhaps with repetitions, such that $P(z)=\alpha_n(z-\lambda_1)\cdots(z-\lambda_n)$.

The number of times $\lambda\in\mathbb{C}$ appears in this expansion is called
the \key{algebraic multiplicity} of the zero.

\subsection{Eigenvalues and Eigenvectors}

\paragraph{D 8.2.1} Given $A\in\mathbb{R}^{n\times n}$, we say
$\lambda\in\mathbb{C}$ is an \key{eigenvalue} of $A$ and
$v\in\mathbb{C}^n\setminus\{0\}$ is an \key{eigenvector} of $A$,
associated with the eigenvalue $\lambda$, when the following holds: $Av=\lambda v$.

We call them an eigenvalue-eigenvector pair. If $\lambda\in\mathbb{R}$, then we
call $\lambda$ a \key{real} eigenvalue, and the associated eigenvalue-eigenvector
pair a \key{real} eigenvalue-eigenvector pair.

\paragraph{L 8.2.3} Let $A\in\mathbb{R}^{n\times n}$.
$\lambda\in\mathbb{R}$ is a (real) eigenvalue of $A$ iff
$\det(A-\lambda I)=0$. A vector $v \in \mathbb{R}^n\setminus\{0\}$ is an eigenvector associated with the
eigenvalue $\lambda$ iff it is a non-zero element of
$\nullsp(A-\lambda I)$.

\paragraph{P 8.2.7} Let $Q\in\mathbb{R}^{n\times n}$ be an orthogonal matrix. If $\lambda\in\mathbb{C}$ is an eigenvalue of $Q$, the $\lvert\lambda\rvert=1$.

\paragraph{L 8.2.8} Let $A \in \mathbb{R}^{n \times n}$. If $(\lambda, v)$ is
an eigenvalue-eigenvector pair, so is $(\overline{\lambda}, \overline{v})$.

\paragraph{P 8.3.1}
\begin{enumerate}
    \item If $\lambda$ and $v$ are an
eigenvalue-eigenvector pair of matrix $A$, then, for $k\geq 1$, $\lambda^k$ and
$v$ are an eigenvalue-eigenvector pair of the matrix $A^k$.

    \item Let $A$ be an invertible matrix. If $\lambda$ and
$v$ are an eigenvalue-eigenvector pair of the matrix $A$, then
$\frac{1}{\lambda}$ and $v$ are an eigenvalue-eigenvector pair of the
matrix $A^{-1}$.
\end{enumerate}

\paragraph{L 8.3.2} Let $A\in\mathbb{R}^{n\times n}$ and let $v_1,\dots,v_k\in\mathbb{R}^n$ be eigenvectors corresponding to eigenvalues $\lambda_1,\dots,\lambda_k\in\mathbb{R}$. If $\lambda_1,\dots,\lambda_k$ are all distinct, the eigenvectors $v_1,\dots,v_k$ are linearly independent.

\paragraph{T 8.3.3} Let $A \in \mathbb{R}^{n\times n}$ with $n$ distinct real eigenvalues (the $n$ zeros of $\det(A-\lambda I)$ are all distinct). Then there is a basis of $\mathbb{R}^n$, $v_1,\dots,v_n$, made up of eigenvectors of $A$.

\paragraph{D 8.3.4} (\key{Characteristic polynomial}) Let $A\in\mathbb{R}^{n\times n}$.
Then $\det(zI-A)$ is a polynomial, in $\lambda$, of degree $n$, which can
according to \textit{C 8.1.3} be factorised as follows:
$\det(zI-A)=(z-\lambda_1)\cdots(z-\lambda_n)$.

This polynomial is called the characteristic polynomial of $A$, and its roots
$\lambda_1,\dots,\lambda_n\in\mathbb{C}$ are the eigenvalues of $A$. The
number of times an eigenvalue shows up is called \key{algebraic multiplicity}
of that eigenvalue.

The \key{trace} of $A$ is defined as $\tr(A) = \sum^n_{i=1} A_{ii}$.

\paragraph{P 8.3.5} Given $A\in\mathbb{R}^{n\times n}$ the
eigenvalues of $A$ are the same as the ones of $A^\top$.

\paragraph{L 8.3.6} Let $A\in\mathbb{R}^{n\times n}$ and $\lambda_1,\dots,\lambda_n$ its $n$ eigenvalues as they show up as zeroes in its characteristic polynomial, then:
\begin{enumerate}
    \item $\tr(A)=\sum_{i=1}^{n}\lambda_i$
    \item $\det(A)=\prod_{i=1}^{n}\lambda_i$
\end{enumerate}

\paragraph{L 8.3.7} For matrices $A,B,C\in\mathbb{R}^{n\times n}$, we
have: $\tr(AB)=\tr(BA)$, $\tr(ABC)=\tr(BCA)=\tr(CAB)$.
